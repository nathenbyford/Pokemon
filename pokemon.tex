% Options for packages loaded elsewhere
\PassOptionsToPackage{unicode}{hyperref}
\PassOptionsToPackage{hyphens}{url}
%
\documentclass[
]{article}
\title{Pokemon Analysis}
\author{Connor Bryson, Nathen Byford, Miguel Iglesias}
\date{10/12/2021}

\usepackage{amsmath,amssymb}
\usepackage{lmodern}
\usepackage{iftex}
\ifPDFTeX
  \usepackage[T1]{fontenc}
  \usepackage[utf8]{inputenc}
  \usepackage{textcomp} % provide euro and other symbols
\else % if luatex or xetex
  \usepackage{unicode-math}
  \defaultfontfeatures{Scale=MatchLowercase}
  \defaultfontfeatures[\rmfamily]{Ligatures=TeX,Scale=1}
\fi
% Use upquote if available, for straight quotes in verbatim environments
\IfFileExists{upquote.sty}{\usepackage{upquote}}{}
\IfFileExists{microtype.sty}{% use microtype if available
  \usepackage[]{microtype}
  \UseMicrotypeSet[protrusion]{basicmath} % disable protrusion for tt fonts
}{}
\makeatletter
\@ifundefined{KOMAClassName}{% if non-KOMA class
  \IfFileExists{parskip.sty}{%
    \usepackage{parskip}
  }{% else
    \setlength{\parindent}{0pt}
    \setlength{\parskip}{6pt plus 2pt minus 1pt}}
}{% if KOMA class
  \KOMAoptions{parskip=half}}
\makeatother
\usepackage{xcolor}
\IfFileExists{xurl.sty}{\usepackage{xurl}}{} % add URL line breaks if available
\IfFileExists{bookmark.sty}{\usepackage{bookmark}}{\usepackage{hyperref}}
\hypersetup{
  pdftitle={Pokemon Analysis},
  pdfauthor={Connor Bryson, Nathen Byford, Miguel Iglesias},
  hidelinks,
  pdfcreator={LaTeX via pandoc}}
\urlstyle{same} % disable monospaced font for URLs
\usepackage[margin=1in]{geometry}
\usepackage{color}
\usepackage{fancyvrb}
\newcommand{\VerbBar}{|}
\newcommand{\VERB}{\Verb[commandchars=\\\{\}]}
\DefineVerbatimEnvironment{Highlighting}{Verbatim}{commandchars=\\\{\}}
% Add ',fontsize=\small' for more characters per line
\usepackage{framed}
\definecolor{shadecolor}{RGB}{248,248,248}
\newenvironment{Shaded}{\begin{snugshade}}{\end{snugshade}}
\newcommand{\AlertTok}[1]{\textcolor[rgb]{0.94,0.16,0.16}{#1}}
\newcommand{\AnnotationTok}[1]{\textcolor[rgb]{0.56,0.35,0.01}{\textbf{\textit{#1}}}}
\newcommand{\AttributeTok}[1]{\textcolor[rgb]{0.77,0.63,0.00}{#1}}
\newcommand{\BaseNTok}[1]{\textcolor[rgb]{0.00,0.00,0.81}{#1}}
\newcommand{\BuiltInTok}[1]{#1}
\newcommand{\CharTok}[1]{\textcolor[rgb]{0.31,0.60,0.02}{#1}}
\newcommand{\CommentTok}[1]{\textcolor[rgb]{0.56,0.35,0.01}{\textit{#1}}}
\newcommand{\CommentVarTok}[1]{\textcolor[rgb]{0.56,0.35,0.01}{\textbf{\textit{#1}}}}
\newcommand{\ConstantTok}[1]{\textcolor[rgb]{0.00,0.00,0.00}{#1}}
\newcommand{\ControlFlowTok}[1]{\textcolor[rgb]{0.13,0.29,0.53}{\textbf{#1}}}
\newcommand{\DataTypeTok}[1]{\textcolor[rgb]{0.13,0.29,0.53}{#1}}
\newcommand{\DecValTok}[1]{\textcolor[rgb]{0.00,0.00,0.81}{#1}}
\newcommand{\DocumentationTok}[1]{\textcolor[rgb]{0.56,0.35,0.01}{\textbf{\textit{#1}}}}
\newcommand{\ErrorTok}[1]{\textcolor[rgb]{0.64,0.00,0.00}{\textbf{#1}}}
\newcommand{\ExtensionTok}[1]{#1}
\newcommand{\FloatTok}[1]{\textcolor[rgb]{0.00,0.00,0.81}{#1}}
\newcommand{\FunctionTok}[1]{\textcolor[rgb]{0.00,0.00,0.00}{#1}}
\newcommand{\ImportTok}[1]{#1}
\newcommand{\InformationTok}[1]{\textcolor[rgb]{0.56,0.35,0.01}{\textbf{\textit{#1}}}}
\newcommand{\KeywordTok}[1]{\textcolor[rgb]{0.13,0.29,0.53}{\textbf{#1}}}
\newcommand{\NormalTok}[1]{#1}
\newcommand{\OperatorTok}[1]{\textcolor[rgb]{0.81,0.36,0.00}{\textbf{#1}}}
\newcommand{\OtherTok}[1]{\textcolor[rgb]{0.56,0.35,0.01}{#1}}
\newcommand{\PreprocessorTok}[1]{\textcolor[rgb]{0.56,0.35,0.01}{\textit{#1}}}
\newcommand{\RegionMarkerTok}[1]{#1}
\newcommand{\SpecialCharTok}[1]{\textcolor[rgb]{0.00,0.00,0.00}{#1}}
\newcommand{\SpecialStringTok}[1]{\textcolor[rgb]{0.31,0.60,0.02}{#1}}
\newcommand{\StringTok}[1]{\textcolor[rgb]{0.31,0.60,0.02}{#1}}
\newcommand{\VariableTok}[1]{\textcolor[rgb]{0.00,0.00,0.00}{#1}}
\newcommand{\VerbatimStringTok}[1]{\textcolor[rgb]{0.31,0.60,0.02}{#1}}
\newcommand{\WarningTok}[1]{\textcolor[rgb]{0.56,0.35,0.01}{\textbf{\textit{#1}}}}
\usepackage{graphicx}
\makeatletter
\def\maxwidth{\ifdim\Gin@nat@width>\linewidth\linewidth\else\Gin@nat@width\fi}
\def\maxheight{\ifdim\Gin@nat@height>\textheight\textheight\else\Gin@nat@height\fi}
\makeatother
% Scale images if necessary, so that they will not overflow the page
% margins by default, and it is still possible to overwrite the defaults
% using explicit options in \includegraphics[width, height, ...]{}
\setkeys{Gin}{width=\maxwidth,height=\maxheight,keepaspectratio}
% Set default figure placement to htbp
\makeatletter
\def\fps@figure{htbp}
\makeatother
\setlength{\emergencystretch}{3em} % prevent overfull lines
\providecommand{\tightlist}{%
  \setlength{\itemsep}{0pt}\setlength{\parskip}{0pt}}
\setcounter{secnumdepth}{-\maxdimen} % remove section numbering
\ifLuaTeX
  \usepackage{selnolig}  % disable illegal ligatures
\fi

\begin{document}
\maketitle

\hypertarget{introduction}{%
\subsection{Introduction}\label{introduction}}

\textless\textless\textless\textless\textless\textless\textless{} HEAD
Pokémon, a Japanese card and video game, revolves around a fantasy world
where people fight each other with creatures they find and domesticate.
Each creature, or Pokémon, can be characterized by certain attributes
from health points (hp), to their typing which includes fire-Pokémon,
water-Pokémon, and grass-Pokémon among others. Our dataset is a CSV that
takes all the defined characteristics of the Pokémon and collects into
one useable file. Though the dataset includes 49 variables, the ones we
are using include: \_\_\_\_\_\_\_\_\_\_\_. ======= Pokémon, a Japanese
card and video game, revolves around a fantasy world where people fight
each other with creatures they find, capture, and then domesticate. Each
creature, or Pokémon, can be characterized by certain attributes
including type (fire, grass, poison etc.), identifiers (legendary vs
non-legendary etc.), and skill points (health, attack, defense etc.).
Our dataset is a CSV that takes all the attributes of every Pokémon and
collects into one usable file. Though the dataset includes 49 variables,
we primarily looked at speed, capture rate, and happiness. We choose
these variables for a variety of reasons. Speed was chosen, because
during the exploratory portion of the project we found that speed and
weight had no significant correlation. So, I {[}Connor{]} decided to
investigate if the developers of the game correlated speed to any other
variables. The lack of relationship between weight and speed made us
distrust the development of the game. It seemed as though the developers
did not put much thought into the values they gave each Pokémon. Given
that we were already looking at speed, I {[}Miguel{]} wanted to see if
the capture rate had any predictors. I {[}Nate{]} decided to do the same
but with the base happiness of the Pokémon. Each portion of the analysis
will be discussed in more detail throughout the paper to give context to
variables and analysis ideas.

\begin{figure}
\centering
\includegraphics{"./Pokedex_Picture"}
\caption{The pokedex entry for Pikachu. Includes info such as type,
height, and weight}
\end{figure}

\begin{quote}
\begin{quote}
\begin{quote}
\begin{quote}
\begin{quote}
\begin{quote}
\begin{quote}
e14794bf5c0251c6de984ddef72b80e3f9e383e3
\end{quote}
\end{quote}
\end{quote}
\end{quote}
\end{quote}
\end{quote}
\end{quote}

\hypertarget{questions-and-findings}{%
\subsection{Questions and Findings}\label{questions-and-findings}}

\hypertarget{speed}{%
\subsubsection{Speed}\label{speed}}

I {[}Connor{]} want to begin with looking into what factors effect the
speed of the Pokemon. This included testing whether real life factors
such as weight and height affected the speed of the Pokemon. At first we
believed they would, but as seen by the graphs below we were wrong.

\begin{Shaded}
\begin{Highlighting}[]
\CommentTok{\#Weight vs Speed (organized by Generation)}
\NormalTok{spdvswgt }\OtherTok{\textless{}{-}} \FunctionTok{ggplot}\NormalTok{(pokemon, }\FunctionTok{aes}\NormalTok{(}\AttributeTok{x =}\NormalTok{weight, }\AttributeTok{y =}\NormalTok{speed, }\AttributeTok{color=}\NormalTok{ gen\_introduced)) }\SpecialCharTok{+}
  \FunctionTok{geom\_point}\NormalTok{()}\SpecialCharTok{+}
  \FunctionTok{labs}\NormalTok{(}\AttributeTok{title=}\StringTok{"Weight vs Speed"}\NormalTok{, }\AttributeTok{x=} \StringTok{"Weight"}\NormalTok{, }\AttributeTok{y=}\StringTok{"Speed"}\NormalTok{)}\SpecialCharTok{+}
  \FunctionTok{xlim}\NormalTok{(}\FunctionTok{c}\NormalTok{(}\DecValTok{0}\NormalTok{, }\FunctionTok{max}\NormalTok{(pokemon}\SpecialCharTok{$}\NormalTok{weight))) }\SpecialCharTok{+} 
  \FunctionTok{ylim}\NormalTok{(}\FunctionTok{c}\NormalTok{(}\DecValTok{0}\NormalTok{, }\FunctionTok{max}\NormalTok{(pokemon}\SpecialCharTok{$}\NormalTok{speed)))}\SpecialCharTok{+}
\NormalTok{  viridis}\SpecialCharTok{::}\FunctionTok{scale\_color\_viridis}\NormalTok{()}
 

\CommentTok{\#Speed vs Height (organized by Generation)}
\NormalTok{spdvshgt }\OtherTok{\textless{}{-}} \FunctionTok{ggplot}\NormalTok{(pokemon, }\FunctionTok{aes}\NormalTok{(}\AttributeTok{x =}\NormalTok{speed, }\AttributeTok{y =}\NormalTok{height, }\AttributeTok{color=}\NormalTok{ gen\_introduced)) }\SpecialCharTok{+}
  \FunctionTok{geom\_point}\NormalTok{()}\SpecialCharTok{+}
  \FunctionTok{labs}\NormalTok{(}\AttributeTok{title=}\StringTok{"Speed vs Height"}\NormalTok{, }\AttributeTok{x=} \StringTok{"Speed"}\NormalTok{, }\AttributeTok{y=}\StringTok{"Height"}\NormalTok{)}\SpecialCharTok{+}
  \FunctionTok{xlim}\NormalTok{(}\FunctionTok{c}\NormalTok{(}\DecValTok{0}\NormalTok{, }\FunctionTok{max}\NormalTok{(pokemon}\SpecialCharTok{$}\NormalTok{speed))) }\SpecialCharTok{+} 
  \FunctionTok{ylim}\NormalTok{(}\FunctionTok{c}\NormalTok{(}\DecValTok{0}\NormalTok{, }\FunctionTok{max}\NormalTok{(pokemon}\SpecialCharTok{$}\NormalTok{height)))}\SpecialCharTok{+}
\NormalTok{  viridis}\SpecialCharTok{::}\FunctionTok{scale\_color\_viridis}\NormalTok{()}\SpecialCharTok{+}
  \FunctionTok{theme}\NormalTok{(}\AttributeTok{plot.title =} \FunctionTok{element\_text}\NormalTok{(}\AttributeTok{hjust =} \FloatTok{0.5}\NormalTok{))}

\NormalTok{spdvswgt}
\end{Highlighting}
\end{Shaded}

\includegraphics{pokemon_files/figure-latex/unnamed-chunk-1-1.pdf}

\begin{Shaded}
\begin{Highlighting}[]
\NormalTok{spdvshgt}
\end{Highlighting}
\end{Shaded}

\includegraphics{pokemon_files/figure-latex/unnamed-chunk-1-2.pdf}
Looking at the graph for weight and speed, we noticed that there are
outliers (on the right side of the graph) that do not make sense for
speed. To see whether this was a mistake or not, we looked at the
generation of each Pokemon and found that the majority of the outliers
came from the most recent generation rather than the earlier generations
when Pokemon's system of attributes were still in the works. In addition
to weight, height did not have an affect on the speed of a Pokemon, thus
defying all of our expectations of this data set.

Afterwards, we wanted to see if the speed of a Pokemon affected its
overall capture rate and whether it is harder to catch a Pokemon if it
has a higher speed or not.

\begin{Shaded}
\begin{Highlighting}[]
\CommentTok{\#Speed vs Capture Rate}
\FunctionTok{ggplot}\NormalTok{(pokemon, }\FunctionTok{aes}\NormalTok{(}\AttributeTok{x =}\NormalTok{speed, }\AttributeTok{y =}\NormalTok{capture\_rate,  }\AttributeTok{color=}\NormalTok{ gen\_introduced)) }\SpecialCharTok{+}
  \FunctionTok{geom\_point}\NormalTok{()}\SpecialCharTok{+}
  \FunctionTok{labs}\NormalTok{(}\AttributeTok{title=}\StringTok{"Speed vs Capture Rate"}\NormalTok{, }\AttributeTok{x=} \StringTok{"Speed"}\NormalTok{, }\AttributeTok{y=}\StringTok{"Capture Rate"}\NormalTok{)}\SpecialCharTok{+}
  \FunctionTok{xlim}\NormalTok{(}\FunctionTok{c}\NormalTok{(}\DecValTok{0}\NormalTok{, }\FunctionTok{max}\NormalTok{(pokemon}\SpecialCharTok{$}\NormalTok{speed))) }\SpecialCharTok{+} 
  \FunctionTok{ylim}\NormalTok{(}\FunctionTok{c}\NormalTok{(}\DecValTok{0}\NormalTok{, }\FunctionTok{max}\NormalTok{(pokemon}\SpecialCharTok{$}\NormalTok{capture\_rate)))}\SpecialCharTok{+}
  \FunctionTok{geom\_smooth}\NormalTok{(}\AttributeTok{se=}\ConstantTok{FALSE}\NormalTok{)}\SpecialCharTok{+}
\NormalTok{  viridis}\SpecialCharTok{::}\FunctionTok{scale\_color\_viridis}\NormalTok{()}\SpecialCharTok{+}
  \FunctionTok{theme}\NormalTok{(}\AttributeTok{plot.title =} \FunctionTok{element\_text}\NormalTok{(}\AttributeTok{hjust =} \FloatTok{0.5}\NormalTok{))}
\end{Highlighting}
\end{Shaded}

\begin{verbatim}
## `geom_smooth()` using method = 'gam' and formula 'y ~ s(x, bs = "cs")'
\end{verbatim}

\includegraphics{pokemon_files/figure-latex/unnamed-chunk-2-1.pdf} The
plot shows a general negative trend as the speed increases, the capture
rate decreases.

Finally with speed, since there was no realistic interpretation of speed
for a given Pokemon, we decided to see if there was a relationship
between a Pokemon's speed of attack and defense.

\begin{Shaded}
\begin{Highlighting}[]
\CommentTok{\#Speed vs Attack }
\NormalTok{spdvsatk }\OtherTok{\textless{}{-}}\FunctionTok{ggplot}\NormalTok{(pokemon, }\FunctionTok{aes}\NormalTok{(}\AttributeTok{x =}\NormalTok{speed, }\AttributeTok{y =}\NormalTok{attack,  }\AttributeTok{color=}\NormalTok{ gen\_introduced)) }\SpecialCharTok{+}
  \FunctionTok{geom\_point}\NormalTok{()}\SpecialCharTok{+}
  \FunctionTok{labs}\NormalTok{(}\AttributeTok{title=}\StringTok{"Speed vs Attack"}\NormalTok{, }\AttributeTok{x=} \StringTok{"Speed"}\NormalTok{, }\AttributeTok{y=}\StringTok{"Attack"}\NormalTok{)}\SpecialCharTok{+}
  \FunctionTok{xlim}\NormalTok{(}\FunctionTok{c}\NormalTok{(}\DecValTok{0}\NormalTok{, }\FunctionTok{max}\NormalTok{(pokemon}\SpecialCharTok{$}\NormalTok{speed))) }\SpecialCharTok{+} 
  \FunctionTok{ylim}\NormalTok{(}\FunctionTok{c}\NormalTok{(}\DecValTok{0}\NormalTok{, }\FunctionTok{max}\NormalTok{(pokemon}\SpecialCharTok{$}\NormalTok{attack)))}\SpecialCharTok{+}
  \FunctionTok{geom\_smooth}\NormalTok{()}\SpecialCharTok{+}
\NormalTok{  viridis}\SpecialCharTok{::}\FunctionTok{scale\_color\_viridis}\NormalTok{()}\SpecialCharTok{+}
  \FunctionTok{theme}\NormalTok{(}\AttributeTok{plot.title =} \FunctionTok{element\_text}\NormalTok{(}\AttributeTok{hjust =} \FloatTok{0.5}\NormalTok{))}

\CommentTok{\#Speed vs Defense}
\NormalTok{spdvsdef }\OtherTok{\textless{}{-}}\FunctionTok{ggplot}\NormalTok{(pokemon, }\FunctionTok{aes}\NormalTok{(}\AttributeTok{x =}\NormalTok{speed, }\AttributeTok{y =}\NormalTok{defense,  }\AttributeTok{color=}\NormalTok{ gen\_introduced)) }\SpecialCharTok{+}
  \FunctionTok{geom\_point}\NormalTok{()}\SpecialCharTok{+}
  \FunctionTok{labs}\NormalTok{(}\AttributeTok{title=}\StringTok{"Speed vs Defense"}\NormalTok{, }\AttributeTok{x=} \StringTok{"Speed"}\NormalTok{, }\AttributeTok{y=}\StringTok{"Defense"}\NormalTok{)}\SpecialCharTok{+}
  \FunctionTok{xlim}\NormalTok{(}\FunctionTok{c}\NormalTok{(}\DecValTok{0}\NormalTok{, }\FunctionTok{max}\NormalTok{(pokemon}\SpecialCharTok{$}\NormalTok{speed))) }\SpecialCharTok{+} 
  \FunctionTok{ylim}\NormalTok{(}\FunctionTok{c}\NormalTok{(}\DecValTok{0}\NormalTok{, }\FunctionTok{max}\NormalTok{(pokemon}\SpecialCharTok{$}\NormalTok{defense)))}\SpecialCharTok{+}
  \FunctionTok{geom\_smooth}\NormalTok{()}\SpecialCharTok{+}
\NormalTok{  viridis}\SpecialCharTok{::}\FunctionTok{scale\_color\_viridis}\NormalTok{()}\SpecialCharTok{+}
  \FunctionTok{theme}\NormalTok{(}\AttributeTok{plot.title =} \FunctionTok{element\_text}\NormalTok{(}\AttributeTok{hjust =} \FloatTok{0.5}\NormalTok{))}

\NormalTok{spdvsatk}\SpecialCharTok{+}\NormalTok{spdvsdef}
\end{Highlighting}
\end{Shaded}

\begin{verbatim}
## `geom_smooth()` using method = 'gam' and formula 'y ~ s(x, bs = "cs")'
## `geom_smooth()` using method = 'gam' and formula 'y ~ s(x, bs = "cs")'
\end{verbatim}

\includegraphics{pokemon_files/figure-latex/unnamed-chunk-3-1.pdf} The
results found that attack tends to have a slight positive correlation
with speed while defense does not.

\hypertarget{capture-rate}{%
\subsubsection{Capture Rate}\label{capture-rate}}

\hypertarget{size-versus-capture-rate}{%
\paragraph{Size versus Capture Rate}\label{size-versus-capture-rate}}

As mentioned earlier, I {[}Miguel{]} wanted to test the game developers
and see if anything predicted capture rate. Capture rate describes how
easy it is to capture a Pokémon found in the wild. The lower the capture
rate, the harder the Pokémon is to capture. For more information on
capture rate, a detailed description can be found at:
\url{https://bulbapedia.bulbagarden.net/wiki/Catch_rate}. Of the
variables in the dataset I first decided to look at the influence of
height, weight, and then BMI. It made intuitive sense to me that bigger
Pokémon would be harder to capture so I decided to test it out.

\begin{Shaded}
\begin{Highlighting}[]
\CommentTok{\#Create a plot that shows the relationship between capture\_rate and BMI}
\CommentTok{\#Limit to not include a lot of empty space}
\NormalTok{BMI }\OtherTok{\textless{}{-}} \FunctionTok{ggplot}\NormalTok{(pokemon\_new, }\FunctionTok{aes}\NormalTok{(BMI, capture\_rate)) }\SpecialCharTok{+}
  \FunctionTok{geom\_jitter}\NormalTok{(}\AttributeTok{alpha =} \FloatTok{0.2}\NormalTok{, }\AttributeTok{color =} \StringTok{"Steel Blue"}\NormalTok{, }\AttributeTok{size =}\NormalTok{ .}\DecValTok{5}\NormalTok{) }\SpecialCharTok{+}
  \FunctionTok{geom\_smooth}\NormalTok{(}\AttributeTok{se =} \ConstantTok{FALSE}\NormalTok{, }\AttributeTok{color =} \StringTok{"Steel blue"}\NormalTok{) }\SpecialCharTok{+}
  \FunctionTok{coord\_cartesian}\NormalTok{(}\AttributeTok{ylim =} \FunctionTok{c}\NormalTok{(}\DecValTok{0}\NormalTok{,}\DecValTok{275}\NormalTok{)) }\SpecialCharTok{+}
  \FunctionTok{labs}\NormalTok{(}\AttributeTok{x =} \StringTok{"Body Mass Index"}\NormalTok{, }\AttributeTok{y =} \StringTok{"Capture Rate"}\NormalTok{)}


\CommentTok{\#Create a plot that shows the relationship between capture\_rate and Height}
\CommentTok{\#Limit included to allow for best view of relationship}
\NormalTok{Height }\OtherTok{\textless{}{-}} \FunctionTok{ggplot}\NormalTok{(pokemon\_new, }\FunctionTok{aes}\NormalTok{(height}\SpecialCharTok{/}\DecValTok{10}\NormalTok{, capture\_rate)) }\SpecialCharTok{+}
  \FunctionTok{geom\_jitter}\NormalTok{(}\AttributeTok{alpha =} \FloatTok{0.2}\NormalTok{, }\AttributeTok{color =} \StringTok{"Steel Blue"}\NormalTok{, }\AttributeTok{size =}\NormalTok{ .}\DecValTok{5}\NormalTok{) }\SpecialCharTok{+}
  \FunctionTok{geom\_smooth}\NormalTok{(}\AttributeTok{se =} \ConstantTok{FALSE}\NormalTok{, }\AttributeTok{color =} \StringTok{"Steel blue"}\NormalTok{) }\SpecialCharTok{+}
  \FunctionTok{coord\_cartesian}\NormalTok{(}\AttributeTok{xlim =} \FunctionTok{c}\NormalTok{(}\DecValTok{0}\NormalTok{,}\DecValTok{5}\NormalTok{), }\AttributeTok{ylim =} \FunctionTok{c}\NormalTok{(}\DecValTok{0}\NormalTok{,}\DecValTok{275}\NormalTok{)) }\SpecialCharTok{+}
  \FunctionTok{labs}\NormalTok{(}\AttributeTok{x =} \StringTok{"Height (m)"}\NormalTok{, }\AttributeTok{y =} \ConstantTok{NULL}\NormalTok{)}
  
\CommentTok{\#Create a plot that shows the relationship between capture\_rate and weight}
\CommentTok{\#Limit included to allow for best view of relationship}

\NormalTok{Weight }\OtherTok{\textless{}{-}} \FunctionTok{ggplot}\NormalTok{(pokemon\_new, }\FunctionTok{aes}\NormalTok{(weight}\SpecialCharTok{/}\DecValTok{10}\NormalTok{, capture\_rate)) }\SpecialCharTok{+} 
  \FunctionTok{geom\_jitter}\NormalTok{(}\AttributeTok{alpha =} \FloatTok{0.2}\NormalTok{, }\AttributeTok{color =} \StringTok{"Steel blue"}\NormalTok{, }\AttributeTok{size =}\NormalTok{ .}\DecValTok{5}\NormalTok{) }\SpecialCharTok{+}
  \FunctionTok{geom\_smooth}\NormalTok{(}\AttributeTok{alpha =} \FloatTok{0.5}\NormalTok{, }\AttributeTok{se =} \ConstantTok{FALSE}\NormalTok{ , }\AttributeTok{color =} \StringTok{"Steel blue"}\NormalTok{) }\SpecialCharTok{+}
  \FunctionTok{coord\_cartesian}\NormalTok{(}\AttributeTok{ylim =} \FunctionTok{c}\NormalTok{(}\DecValTok{0}\NormalTok{,}\DecValTok{275}\NormalTok{)) }\SpecialCharTok{+}
  \FunctionTok{labs}\NormalTok{(}\AttributeTok{x =} \StringTok{"Weight (kg)"}\NormalTok{, }\AttributeTok{y =} \StringTok{"Capture Rate"}\NormalTok{)}


\NormalTok{(Weight }\SpecialCharTok{|}\NormalTok{ Height)}
\end{Highlighting}
\end{Shaded}

\begin{verbatim}
## `geom_smooth()` using method = 'gam' and formula 'y ~ s(x, bs = "cs")'
## `geom_smooth()` using method = 'gam' and formula 'y ~ s(x, bs = "cs")'
\end{verbatim}

\includegraphics{pokemon_files/figure-latex/unnamed-chunk-4-1.pdf}

\begin{Shaded}
\begin{Highlighting}[]
\NormalTok{BMI}
\end{Highlighting}
\end{Shaded}

\begin{verbatim}
## `geom_smooth()` using method = 'gam' and formula 'y ~ s(x, bs = "cs")'
\end{verbatim}

\includegraphics{pokemon_files/figure-latex/unnamed-chunk-4-2.pdf}

As can be seen, size does have an influence on the rate of capture. The
taller and heavier Pokémon both have decreased rates of capture.
Interestingly enough, however, when the two are put together as with
BMI, the relationship goes away. To further understand the complex
relationship between size and capture rate, more analysis would be
required, however, the preliminary look suggests that size does have
some affect on capture rate. \#\#\#\# Other variables of interest

After learning more about the relationship between size and capture
rate, I {[}Miguel{]} decided to see how other variables may affect the
capture rate. The first one I examined was the shape of the Pokémon.
Shapes include armor, quadraped, upright and more.

\begin{Shaded}
\begin{Highlighting}[]
\NormalTok{p1 }\OtherTok{\textless{}{-}} \FunctionTok{ggplot}\NormalTok{(pokemon\_new) }\SpecialCharTok{+}
  \FunctionTok{geom\_boxplot}\NormalTok{(}\FunctionTok{aes}\NormalTok{(}\FunctionTok{reorder}\NormalTok{(shape, capture\_rate, }\AttributeTok{FUN =}\NormalTok{ median), capture\_rate)) }\SpecialCharTok{+}
  \FunctionTok{labs}\NormalTok{(}\AttributeTok{x =} \StringTok{"Shape"}\NormalTok{, }\AttributeTok{y =} \StringTok{"Capture Rate"}\NormalTok{)}

\NormalTok{p1}
\end{Highlighting}
\end{Shaded}

\includegraphics{pokemon_files/figure-latex/unnamed-chunk-5-1.pdf}

As seen in the plot above, shape has little influence on capture rate
except for in the cases of Legs and Armor. What may be surprising,
however, is that armored Pokémon are easier to capture than non-armored
Pokémon.

Next, I looked at four of the boolean variables from the dataset. I
wanted to see if how capture rate was influenced by Pokémon being
Legendary, Mythical, genderless, and babies. One might expect mythical
and legendary Pokémon to be harder to catch, babies easier to catch, and
genderless Pokémon to have about equal rates to the gendered ones.

\begin{verbatim}
p2 <- ggplot(pokemon_new, aes(mythical, capture_rate)) + 
  geom_boxplot() +
  labs(x = "Mythical", y = "Capture Rate")

p3 <- ggplot(pokemon_new, aes(legendary, capture_rate)) + 
  geom_boxplot() +
  labs(x = "Legendary", y = NULL)

p4 <- ggplot(pokemon_new, aes(genderless, capture_rate)) + 
  geom_boxplot() +
  labs(x = "Genderless", y = NULL)

p5 <- ggplot(pokemon_new, aes(baby_pokemon, capture_rate)) + 
  geom_boxplot() +
  labs(x = "Baby Pokemon", y = NULL)

p2 | p3 | p4 |p5
  
\end{verbatim}

As expected, legendary and mythical Pokémon are more difficult to catch,
and babies are easier to catch. What is surprsing, however, is that
genderless Pokémon are also harder to catch. This could be due to a
confounding variable, however, it is more likely that the game
developers did not put that much thought into this. In fact, a
preliminary look for confounding variables suggested that the genderless
variable is not confounded by the legendary or mythical variable, both
of which might have explained the increased difficulty of capture for
genderless Pokémon.

Lastly, I looked at how the percentage that each Pokémon is female
influences capture rate. Within the game, Pokémon can be bred to make
new Pokémon. Due to the presence of Pokémon breeding, Pokémon gender
must also exist and each Pokémon has a set probability of being female
when found in the wild. Some Pokémon have a female rate of 12.5\%,
whereas others are as high as 100\%. This does not mean a Pokémon can be
12.5\% female and 87.5\% male, merely that within the whole population
12.5\% of the Pokémon are female. Genderless Pokémon have a female rate
of 0.

\begin{Shaded}
\begin{Highlighting}[]
\NormalTok{p6 }\OtherTok{\textless{}{-}} \FunctionTok{ggplot}\NormalTok{(pokemon\_new, }\FunctionTok{aes}\NormalTok{(}\FunctionTok{as.factor}\NormalTok{(female\_rate}\SpecialCharTok{*}\DecValTok{100}\NormalTok{), capture\_rate)) }\SpecialCharTok{+}
  \FunctionTok{geom\_boxplot}\NormalTok{() }\SpecialCharTok{+}
  \FunctionTok{labs}\NormalTok{(}\AttributeTok{x =} \StringTok{"Percent Female"}\NormalTok{, }\AttributeTok{y =} \StringTok{"Capture Rate"}\NormalTok{)}

\NormalTok{p6}
\end{Highlighting}
\end{Shaded}

\includegraphics{pokemon_files/figure-latex/unnamed-chunk-6-1.pdf}

To avoid any confusion, it must be understood that within each Pokémon,
the males and females have each capture rate. Though, as it turns out,
the female rate does influence the capture rate. Generally speaking, the
Pokémon with high female rates are easier to capture. Though the 0\%
female category may be lower due to the addition of the genderless
Pokémon, the 12.5\% and 25\% catergories are also low, especially
compared to the 87.5\% category. Why male-dominant Pokémon are harder to
capture than female-dominant Pokémon is a question for another day.

\hypertarget{happiness}{%
\subsubsection{Happiness}\label{happiness}}

Lastly we wanted to see what factors effected the happiness of the
Pokemon. This variable was just something we thought would be fun to
look into. Our main questions were what other variables effect the
Pokemon's happiness. We began with looking at if the Pokemon is a
legendary and if it's a mythical Pokemon.

To get a grasp of what effects different variables had on the happiness
of the Pokemon we took a look at how different characteristics of the
Pokemon had an effect. Bellow are three violin plots looking at
mythical, legendary, and baby Pokemon and comparing them to Pokemon
without that attribute based on happiness. It is fairly obvious that
mythical and legendary Pokemon tend to be less happy, I thinkk that
could be that they are tougher pokemon so when they are captured they
are not happy about it. These Pokemon like to have free reign and not be
controlled by trainers. While the Baby Pokemon look to be as happy or
more happy than most other pokemon.

\begin{Shaded}
\begin{Highlighting}[]
\NormalTok{v1 }\OtherTok{\textless{}{-}}\NormalTok{ pokemon\_new }\SpecialCharTok{|}\ErrorTok{\textgreater{}} \FunctionTok{ggplot}\NormalTok{(}\FunctionTok{aes}\NormalTok{(mythical, base\_happiness)) }\SpecialCharTok{+}
  \FunctionTok{geom\_violin}\NormalTok{() }\SpecialCharTok{+}
  \FunctionTok{labs}\NormalTok{(}\AttributeTok{title =} \StringTok{"Mythical Pokemon"}\NormalTok{, }\AttributeTok{x =} \StringTok{"Mythical"}\NormalTok{, }\AttributeTok{y =} \StringTok{"Happiness"}\NormalTok{)}

\NormalTok{v2 }\OtherTok{\textless{}{-}}\NormalTok{ pokemon\_new }\SpecialCharTok{|}\ErrorTok{\textgreater{}} \FunctionTok{ggplot}\NormalTok{(}\FunctionTok{aes}\NormalTok{(legendary, base\_happiness)) }\SpecialCharTok{+}
  \FunctionTok{geom\_violin}\NormalTok{() }\SpecialCharTok{+}
  \FunctionTok{labs}\NormalTok{(}\AttributeTok{title =} \StringTok{"Legendary"}\NormalTok{, }\AttributeTok{x =} \StringTok{"Legendary"}\NormalTok{, }\AttributeTok{y =} \ConstantTok{NULL}\NormalTok{)}

\NormalTok{v3 }\OtherTok{\textless{}{-}}\NormalTok{ pokemon\_new }\SpecialCharTok{|}\ErrorTok{\textgreater{}} \FunctionTok{ggplot}\NormalTok{(}\FunctionTok{aes}\NormalTok{(baby\_pokemon, base\_happiness)) }\SpecialCharTok{+}
  \FunctionTok{geom\_violin}\NormalTok{() }\SpecialCharTok{+}
  \FunctionTok{labs}\NormalTok{(}\AttributeTok{title =} \StringTok{"Baby Pokemon"}\NormalTok{, }\AttributeTok{x =} \StringTok{"Baby Pokemon"}\NormalTok{, }\AttributeTok{y =} \ConstantTok{NULL}\NormalTok{)}

\NormalTok{(v1 }\SpecialCharTok{+}\NormalTok{ v2 }\SpecialCharTok{+}\NormalTok{ v3)}
\end{Highlighting}
\end{Shaded}

\includegraphics{pokemon_files/figure-latex/unnamed-chunk-7-1.pdf}

We also looked at how the Pokemon primary type effects it's happiness.
Looking at the mosaic plot below we can see that there are different
distributions of happiness bases upon primary type. Some observations to
be made is that the most unhappy pokemon types are dragon, dark, and
steel. Also the majority of Pokemon appear to be neutral.

\begin{Shaded}
\begin{Highlighting}[]
\NormalTok{m1 }\OtherTok{\textless{}{-}}\NormalTok{ pokemon\_new }\SpecialCharTok{|}\ErrorTok{\textgreater{}} \FunctionTok{ggplot}\NormalTok{() }\SpecialCharTok{+}
  \FunctionTok{geom\_mosaic}\NormalTok{(}\FunctionTok{aes}\NormalTok{(}\AttributeTok{x =} \FunctionTok{product}\NormalTok{(happiness, primary), }\AttributeTok{fill =}\NormalTok{ happiness)) }\SpecialCharTok{+} 
  \FunctionTok{labs}\NormalTok{(}\AttributeTok{title =} \StringTok{"Pokemon Happiness by Primary Type"}\NormalTok{, }\AttributeTok{x =} \StringTok{"Primary Type"}\NormalTok{, }
       \AttributeTok{y =} \StringTok{"Happiness"}\NormalTok{) }\SpecialCharTok{+}
  \FunctionTok{theme}\NormalTok{(}
    \AttributeTok{panel.grid.major.y =} \FunctionTok{element\_blank}\NormalTok{(),}
    \AttributeTok{axis.text.y =} \FunctionTok{element\_blank}\NormalTok{(),}
    \AttributeTok{axis.ticks.y =} \FunctionTok{element\_blank}\NormalTok{(),}
    \AttributeTok{axis.text.x.bottom =} \FunctionTok{element\_text}\NormalTok{(}\AttributeTok{angle =} \DecValTok{60}\NormalTok{, }\AttributeTok{vjust =} \DecValTok{1}\NormalTok{, }\AttributeTok{hjust =} \DecValTok{1}\NormalTok{)}
\NormalTok{  )}

\NormalTok{m1}
\end{Highlighting}
\end{Shaded}

\includegraphics{pokemon_files/figure-latex/unnamed-chunk-8-1.pdf}

We were also curious if the female rate effected happiness of the
Pokemon. Converting the happiness variable to a likert scale we can
easily see the amount of data points in each bin of female rate in the
following plot. As seen in the plots before there is a smaller amount of
Pokemon with a high female rate. Notice how it appears that depending of
the female rate there are different levels of happiness that are
observed. What stood out to us what that Pokemon with a female rate of
0.25 or 0.875 are only observed with neutral happiness and there are no
very unhappy Pokemon with a rate of 1. There is no apparent correlation
between happiness and female rate.

\begin{Shaded}
\begin{Highlighting}[]
\NormalTok{pokemon\_new }\SpecialCharTok{|}\ErrorTok{\textgreater{}} \FunctionTok{ggplot}\NormalTok{(}\FunctionTok{aes}\NormalTok{(}\FunctionTok{as.factor}\NormalTok{(female\_rate), happiness)) }\SpecialCharTok{+}
  \FunctionTok{geom\_bin2d}\NormalTok{(}\AttributeTok{alpha =}\NormalTok{ .}\DecValTok{9}\NormalTok{) }\SpecialCharTok{+}
  \FunctionTok{scale\_fill\_viridis\_c}\NormalTok{() }\SpecialCharTok{+}
  \FunctionTok{labs}\NormalTok{(}\AttributeTok{title =} \StringTok{"Happiness of Pokemon by Female Rate"}\NormalTok{, }\AttributeTok{x =} \StringTok{"Female Rate"}\NormalTok{, }\AttributeTok{y =} \StringTok{"Happiness"}\NormalTok{) }\SpecialCharTok{+}
  \FunctionTok{theme}\NormalTok{(}\AttributeTok{panel.grid.major.x =} \FunctionTok{element\_blank}\NormalTok{())}
\end{Highlighting}
\end{Shaded}

\includegraphics{pokemon_files/figure-latex/unnamed-chunk-9-1.pdf}

\end{document}
