% Options for packages loaded elsewhere
\PassOptionsToPackage{unicode}{hyperref}
\PassOptionsToPackage{hyphens}{url}
%
\documentclass[
]{article}
\title{Pokemon Analysis}
\author{Connor Bryson, Nathen Byford, Miguel Iglesias}
\date{10/12/2021}

\usepackage{amsmath,amssymb}
\usepackage{lmodern}
\usepackage{iftex}
\ifPDFTeX
  \usepackage[T1]{fontenc}
  \usepackage[utf8]{inputenc}
  \usepackage{textcomp} % provide euro and other symbols
\else % if luatex or xetex
  \usepackage{unicode-math}
  \defaultfontfeatures{Scale=MatchLowercase}
  \defaultfontfeatures[\rmfamily]{Ligatures=TeX,Scale=1}
\fi
% Use upquote if available, for straight quotes in verbatim environments
\IfFileExists{upquote.sty}{\usepackage{upquote}}{}
\IfFileExists{microtype.sty}{% use microtype if available
  \usepackage[]{microtype}
  \UseMicrotypeSet[protrusion]{basicmath} % disable protrusion for tt fonts
}{}
\makeatletter
\@ifundefined{KOMAClassName}{% if non-KOMA class
  \IfFileExists{parskip.sty}{%
    \usepackage{parskip}
  }{% else
    \setlength{\parindent}{0pt}
    \setlength{\parskip}{6pt plus 2pt minus 1pt}}
}{% if KOMA class
  \KOMAoptions{parskip=half}}
\makeatother
\usepackage{xcolor}
\IfFileExists{xurl.sty}{\usepackage{xurl}}{} % add URL line breaks if available
\IfFileExists{bookmark.sty}{\usepackage{bookmark}}{\usepackage{hyperref}}
\hypersetup{
  pdftitle={Pokemon Analysis},
  pdfauthor={Connor Bryson, Nathen Byford, Miguel Iglesias},
  hidelinks,
  pdfcreator={LaTeX via pandoc}}
\urlstyle{same} % disable monospaced font for URLs
\usepackage[margin=1in]{geometry}
\usepackage{color}
\usepackage{fancyvrb}
\newcommand{\VerbBar}{|}
\newcommand{\VERB}{\Verb[commandchars=\\\{\}]}
\DefineVerbatimEnvironment{Highlighting}{Verbatim}{commandchars=\\\{\}}
% Add ',fontsize=\small' for more characters per line
\usepackage{framed}
\definecolor{shadecolor}{RGB}{248,248,248}
\newenvironment{Shaded}{\begin{snugshade}}{\end{snugshade}}
\newcommand{\AlertTok}[1]{\textcolor[rgb]{0.94,0.16,0.16}{#1}}
\newcommand{\AnnotationTok}[1]{\textcolor[rgb]{0.56,0.35,0.01}{\textbf{\textit{#1}}}}
\newcommand{\AttributeTok}[1]{\textcolor[rgb]{0.77,0.63,0.00}{#1}}
\newcommand{\BaseNTok}[1]{\textcolor[rgb]{0.00,0.00,0.81}{#1}}
\newcommand{\BuiltInTok}[1]{#1}
\newcommand{\CharTok}[1]{\textcolor[rgb]{0.31,0.60,0.02}{#1}}
\newcommand{\CommentTok}[1]{\textcolor[rgb]{0.56,0.35,0.01}{\textit{#1}}}
\newcommand{\CommentVarTok}[1]{\textcolor[rgb]{0.56,0.35,0.01}{\textbf{\textit{#1}}}}
\newcommand{\ConstantTok}[1]{\textcolor[rgb]{0.00,0.00,0.00}{#1}}
\newcommand{\ControlFlowTok}[1]{\textcolor[rgb]{0.13,0.29,0.53}{\textbf{#1}}}
\newcommand{\DataTypeTok}[1]{\textcolor[rgb]{0.13,0.29,0.53}{#1}}
\newcommand{\DecValTok}[1]{\textcolor[rgb]{0.00,0.00,0.81}{#1}}
\newcommand{\DocumentationTok}[1]{\textcolor[rgb]{0.56,0.35,0.01}{\textbf{\textit{#1}}}}
\newcommand{\ErrorTok}[1]{\textcolor[rgb]{0.64,0.00,0.00}{\textbf{#1}}}
\newcommand{\ExtensionTok}[1]{#1}
\newcommand{\FloatTok}[1]{\textcolor[rgb]{0.00,0.00,0.81}{#1}}
\newcommand{\FunctionTok}[1]{\textcolor[rgb]{0.00,0.00,0.00}{#1}}
\newcommand{\ImportTok}[1]{#1}
\newcommand{\InformationTok}[1]{\textcolor[rgb]{0.56,0.35,0.01}{\textbf{\textit{#1}}}}
\newcommand{\KeywordTok}[1]{\textcolor[rgb]{0.13,0.29,0.53}{\textbf{#1}}}
\newcommand{\NormalTok}[1]{#1}
\newcommand{\OperatorTok}[1]{\textcolor[rgb]{0.81,0.36,0.00}{\textbf{#1}}}
\newcommand{\OtherTok}[1]{\textcolor[rgb]{0.56,0.35,0.01}{#1}}
\newcommand{\PreprocessorTok}[1]{\textcolor[rgb]{0.56,0.35,0.01}{\textit{#1}}}
\newcommand{\RegionMarkerTok}[1]{#1}
\newcommand{\SpecialCharTok}[1]{\textcolor[rgb]{0.00,0.00,0.00}{#1}}
\newcommand{\SpecialStringTok}[1]{\textcolor[rgb]{0.31,0.60,0.02}{#1}}
\newcommand{\StringTok}[1]{\textcolor[rgb]{0.31,0.60,0.02}{#1}}
\newcommand{\VariableTok}[1]{\textcolor[rgb]{0.00,0.00,0.00}{#1}}
\newcommand{\VerbatimStringTok}[1]{\textcolor[rgb]{0.31,0.60,0.02}{#1}}
\newcommand{\WarningTok}[1]{\textcolor[rgb]{0.56,0.35,0.01}{\textbf{\textit{#1}}}}
\usepackage{graphicx}
\makeatletter
\def\maxwidth{\ifdim\Gin@nat@width>\linewidth\linewidth\else\Gin@nat@width\fi}
\def\maxheight{\ifdim\Gin@nat@height>\textheight\textheight\else\Gin@nat@height\fi}
\makeatother
% Scale images if necessary, so that they will not overflow the page
% margins by default, and it is still possible to overwrite the defaults
% using explicit options in \includegraphics[width, height, ...]{}
\setkeys{Gin}{width=\maxwidth,height=\maxheight,keepaspectratio}
% Set default figure placement to htbp
\makeatletter
\def\fps@figure{htbp}
\makeatother
\setlength{\emergencystretch}{3em} % prevent overfull lines
\providecommand{\tightlist}{%
  \setlength{\itemsep}{0pt}\setlength{\parskip}{0pt}}
\setcounter{secnumdepth}{-\maxdimen} % remove section numbering
\ifLuaTeX
  \usepackage{selnolig}  % disable illegal ligatures
\fi

\begin{document}
\maketitle

\hypertarget{introduction}{%
\subsection{Introduction}\label{introduction}}

Pokémon, a Japanese card and video game, revolves around a fantasy world
where people fight each other with creatures they find and domesticate.
Each creature, or Pokémon, can be characterized by certain attributes
from health points (hp), to their typing which includes fire-Pokémon,
water-Pokémon, and grass-Pokémon among others. Our dataset is a CSV that
takes all the defined characteristics of the Pokémon and collects into
one useable file. Though the dataset includes 49 variables, the ones we
are using include: \_\_\_\_\_\_\_\_\_\_\_. One variable of interest,
speed, is noteworthy because during the exploratory period of this
project, we found that there was no relationship between speed and
weight. This does not make intuitive sense, so we sought to identify if
any other variable could correlate with speed.

\hypertarget{questions-and-findings}{%
\subsection{Questions and Findings}\label{questions-and-findings}}

\hypertarget{speed}{%
\subsubsection{Speed}\label{speed}}

We want to begin with looking into what factors effect the speed of the
Pokemon.

\hypertarget{capture-rate}{%
\subsubsection{Capture Rate}\label{capture-rate}}

We also wanted to look into how capture rate is effected by variables
such as height, weight, gender, and legendary. Capture rate describes
how easy it is to capture a Pokémon found in the wild. The lower the
capture rate, the harder the Pokémon is to capture. Initially our belief
was that the larger the Pokémon, the harder it would be to capture so we
decided to compare capture rate to BMI, Height and Weight. We then
wanted to see how certain categorical variables affected capture rate in
the hopes of finding a pattern in how the game developers determined
capture rate.

\begin{Shaded}
\begin{Highlighting}[]
\CommentTok{\#Create a plot that shows the relationship between capture\_rate and BMI}
\CommentTok{\#Limit to not include a lot of empty space}
\NormalTok{BMI }\OtherTok{\textless{}{-}} \FunctionTok{ggplot}\NormalTok{(pokemon\_new, }\FunctionTok{aes}\NormalTok{(BMI, capture\_rate)) }\SpecialCharTok{+}
  \FunctionTok{geom\_jitter}\NormalTok{(}\AttributeTok{alpha =} \FloatTok{0.2}\NormalTok{, }\AttributeTok{color =} \StringTok{"Steel Blue"}\NormalTok{, }\AttributeTok{size =}\NormalTok{ .}\DecValTok{5}\NormalTok{) }\SpecialCharTok{+}
  \FunctionTok{geom\_smooth}\NormalTok{(}\AttributeTok{se =} \ConstantTok{FALSE}\NormalTok{) }\SpecialCharTok{+}
  \FunctionTok{coord\_cartesian}\NormalTok{(}\AttributeTok{ylim =} \FunctionTok{c}\NormalTok{(}\DecValTok{0}\NormalTok{,}\DecValTok{275}\NormalTok{)) }\SpecialCharTok{+}
  \FunctionTok{labs}\NormalTok{(}\AttributeTok{x =} \StringTok{"Body Mass Index"}\NormalTok{, }\AttributeTok{y =} \StringTok{"Capture Rate"}\NormalTok{)}


\CommentTok{\#Create a plot that shows the relationship between capture\_rate and Height}
\CommentTok{\#Limit included to allow for best view of relationship}
\NormalTok{Height }\OtherTok{\textless{}{-}} \FunctionTok{ggplot}\NormalTok{(pokemon\_new, }\FunctionTok{aes}\NormalTok{(height}\SpecialCharTok{/}\DecValTok{10}\NormalTok{, capture\_rate)) }\SpecialCharTok{+}
  \FunctionTok{geom\_jitter}\NormalTok{(}\AttributeTok{alpha =} \FloatTok{0.2}\NormalTok{, }\AttributeTok{color =} \StringTok{"Steel Blue"}\NormalTok{, }\AttributeTok{size =}\NormalTok{ .}\DecValTok{5}\NormalTok{) }\SpecialCharTok{+}
  \FunctionTok{geom\_smooth}\NormalTok{(}\AttributeTok{se =} \ConstantTok{FALSE}\NormalTok{) }\SpecialCharTok{+}
  \FunctionTok{coord\_cartesian}\NormalTok{(}\AttributeTok{xlim =} \FunctionTok{c}\NormalTok{(}\DecValTok{0}\NormalTok{,}\DecValTok{5}\NormalTok{), }\AttributeTok{ylim =} \FunctionTok{c}\NormalTok{(}\DecValTok{0}\NormalTok{,}\DecValTok{275}\NormalTok{)) }\SpecialCharTok{+}
  \FunctionTok{labs}\NormalTok{(}\AttributeTok{x =} \StringTok{"Height (m)"}\NormalTok{, }\AttributeTok{y =} \ConstantTok{NULL}\NormalTok{)}
  
\CommentTok{\#Create a plot that shows the relationship between capture\_rate and weight}
\CommentTok{\#Limit included to allow for best view of relationship}

\NormalTok{Weight }\OtherTok{\textless{}{-}} \FunctionTok{ggplot}\NormalTok{(pokemon\_new, }\FunctionTok{aes}\NormalTok{(weight}\SpecialCharTok{/}\DecValTok{10}\NormalTok{, capture\_rate)) }\SpecialCharTok{+} 
  \FunctionTok{geom\_jitter}\NormalTok{(}\AttributeTok{alpha =} \FloatTok{0.2}\NormalTok{, }\AttributeTok{color =} \StringTok{"Steel blue"}\NormalTok{, }\AttributeTok{size =}\NormalTok{ .}\DecValTok{5}\NormalTok{) }\SpecialCharTok{+}
  \FunctionTok{geom\_smooth}\NormalTok{(}\AttributeTok{alpha =} \FloatTok{0.5}\NormalTok{, }\AttributeTok{se =} \ConstantTok{FALSE}\NormalTok{) }\SpecialCharTok{+}
  \FunctionTok{coord\_cartesian}\NormalTok{(}\AttributeTok{ylim =} \FunctionTok{c}\NormalTok{(}\DecValTok{0}\NormalTok{,}\DecValTok{275}\NormalTok{)) }\SpecialCharTok{+}
  \FunctionTok{labs}\NormalTok{(}\AttributeTok{x =} \StringTok{"Weight (kg)"}\NormalTok{, }\AttributeTok{y =} \StringTok{"Capture Rate"}\NormalTok{)}


\NormalTok{BMI }\SpecialCharTok{/}\NormalTok{ (Weight }\SpecialCharTok{|}\NormalTok{ Height)}
\end{Highlighting}
\end{Shaded}

\begin{verbatim}
## `geom_smooth()` using method = 'gam' and formula 'y ~ s(x, bs = "cs")'
## `geom_smooth()` using method = 'gam' and formula 'y ~ s(x, bs = "cs")'
## `geom_smooth()` using method = 'gam' and formula 'y ~ s(x, bs = "cs")'
\end{verbatim}

\includegraphics{pokemon_files/figure-latex/unnamed-chunk-1-1.pdf}

\begin{Shaded}
\begin{Highlighting}[]
\NormalTok{p1 }\OtherTok{\textless{}{-}} \FunctionTok{ggplot}\NormalTok{(pokemon\_new) }\SpecialCharTok{+}
  \FunctionTok{geom\_boxplot}\NormalTok{(}\FunctionTok{aes}\NormalTok{(}\FunctionTok{reorder}\NormalTok{(shape, capture\_rate, }\AttributeTok{FUN =}\NormalTok{ median), capture\_rate)) }\SpecialCharTok{+}
  \FunctionTok{labs}\NormalTok{(}\AttributeTok{x =} \StringTok{"Shape"}\NormalTok{, }\AttributeTok{y =} \StringTok{"Capture Rate"}\NormalTok{)}

\NormalTok{p2 }\OtherTok{\textless{}{-}} \FunctionTok{ggplot}\NormalTok{(pokemon\_new, }\FunctionTok{aes}\NormalTok{(mythical, capture\_rate)) }\SpecialCharTok{+} 
  \FunctionTok{geom\_boxplot}\NormalTok{() }\SpecialCharTok{+}
  \FunctionTok{labs}\NormalTok{(}\AttributeTok{x =} \StringTok{"Mythical"}\NormalTok{, }\AttributeTok{y =} \StringTok{"Capture Rate"}\NormalTok{)}

\NormalTok{p3 }\OtherTok{\textless{}{-}} \FunctionTok{ggplot}\NormalTok{(pokemon\_new, }\FunctionTok{aes}\NormalTok{(legendary, capture\_rate)) }\SpecialCharTok{+} 
  \FunctionTok{geom\_boxplot}\NormalTok{() }\SpecialCharTok{+}
  \FunctionTok{labs}\NormalTok{(}\AttributeTok{x =} \StringTok{"Legendary"}\NormalTok{, }\AttributeTok{y =} \ConstantTok{NULL}\NormalTok{)}

\NormalTok{p4 }\OtherTok{\textless{}{-}} \FunctionTok{ggplot}\NormalTok{(pokemon\_new, }\FunctionTok{aes}\NormalTok{(genderless, capture\_rate)) }\SpecialCharTok{+} 
  \FunctionTok{geom\_boxplot}\NormalTok{() }\SpecialCharTok{+}
  \FunctionTok{labs}\NormalTok{(}\AttributeTok{x =} \StringTok{"Genderless"}\NormalTok{, }\AttributeTok{y =} \ConstantTok{NULL}\NormalTok{)}

\NormalTok{p5 }\OtherTok{\textless{}{-}} \FunctionTok{ggplot}\NormalTok{(pokemon\_new, }\FunctionTok{aes}\NormalTok{(baby\_pokemon, capture\_rate)) }\SpecialCharTok{+} 
  \FunctionTok{geom\_boxplot}\NormalTok{() }\SpecialCharTok{+}
  \FunctionTok{labs}\NormalTok{(}\AttributeTok{x =} \StringTok{"Baby Pokemon"}\NormalTok{, }\AttributeTok{y =} \ConstantTok{NULL}\NormalTok{)}
 
\NormalTok{p6 }\OtherTok{\textless{}{-}} \FunctionTok{ggplot}\NormalTok{(pokemon\_new, }\FunctionTok{aes}\NormalTok{(}\FunctionTok{as.factor}\NormalTok{(female\_rate}\SpecialCharTok{*}\DecValTok{100}\NormalTok{), capture\_rate)) }\SpecialCharTok{+}
  \FunctionTok{geom\_boxplot}\NormalTok{() }\SpecialCharTok{+}
  \FunctionTok{labs}\NormalTok{(}\AttributeTok{x =} \StringTok{"Percent Female"}\NormalTok{, }\AttributeTok{y =} \StringTok{"Capture Rate"}\NormalTok{)}
             
\NormalTok{(p2 }\SpecialCharTok{|}\NormalTok{ p3 }\SpecialCharTok{|}\NormalTok{ p4}\SpecialCharTok{|}\NormalTok{ p5) }\SpecialCharTok{/}\NormalTok{ p6}
\end{Highlighting}
\end{Shaded}

\includegraphics{pokemon_files/figure-latex/unnamed-chunk-1-2.pdf}

\hypertarget{happiness}{%
\subsubsection{Happiness}\label{happiness}}

Lastly we wanted to see what factors effected the happiness of the
Pokemon. This variable was just something we thought would be fun to
look into. Our main questions were what other variables effect the
Pokemon's happiness. We began with looking at if the Pokemon is a
legendary and if it's a mythical Pokemon.

To get a grasp of what effects different variables had on the happiness
of the Pokemon we took a look at how different characteristics of the
Pokemon had an effect. Bellow are three violin plots looking at
mythical, legendary, and baby Pokemon and comparing them to Pokemon
without that attribute based on happiness. It is fairly obvious that
mythical and legendary Pokemon tend to be less happy, I thinkk that
could be that they are tougher pokemon so when they are captured they
are not happy about it. These Pokemon like to have free reign and not be
controlled by trainers. While the Baby Pokemon look to be as happy or
more happy than most other pokemon.

\begin{Shaded}
\begin{Highlighting}[]
\NormalTok{v1 }\OtherTok{\textless{}{-}}\NormalTok{ pokemon\_new }\SpecialCharTok{|}\ErrorTok{\textgreater{}} \FunctionTok{ggplot}\NormalTok{(}\FunctionTok{aes}\NormalTok{(mythical, base\_happiness)) }\SpecialCharTok{+}
  \FunctionTok{geom\_violin}\NormalTok{() }\SpecialCharTok{+}
  \FunctionTok{labs}\NormalTok{(}\AttributeTok{title =} \StringTok{"Mythical Pokemon"}\NormalTok{, }\AttributeTok{x =} \StringTok{"Mythical"}\NormalTok{, }\AttributeTok{y =} \StringTok{"Happiness"}\NormalTok{)}

\NormalTok{v2 }\OtherTok{\textless{}{-}}\NormalTok{ pokemon\_new }\SpecialCharTok{|}\ErrorTok{\textgreater{}} \FunctionTok{ggplot}\NormalTok{(}\FunctionTok{aes}\NormalTok{(legendary, base\_happiness)) }\SpecialCharTok{+}
  \FunctionTok{geom\_violin}\NormalTok{() }\SpecialCharTok{+}
  \FunctionTok{labs}\NormalTok{(}\AttributeTok{title =} \StringTok{"Legendary"}\NormalTok{, }\AttributeTok{x =} \StringTok{"Legendary"}\NormalTok{, }\AttributeTok{y =} \ConstantTok{NULL}\NormalTok{)}

\NormalTok{v3 }\OtherTok{\textless{}{-}}\NormalTok{ pokemon\_new }\SpecialCharTok{|}\ErrorTok{\textgreater{}} \FunctionTok{ggplot}\NormalTok{(}\FunctionTok{aes}\NormalTok{(baby\_pokemon, base\_happiness)) }\SpecialCharTok{+}
  \FunctionTok{geom\_violin}\NormalTok{() }\SpecialCharTok{+}
  \FunctionTok{labs}\NormalTok{(}\AttributeTok{title =} \StringTok{"Baby Pokemon"}\NormalTok{, }\AttributeTok{x =} \StringTok{"Baby Pokemon"}\NormalTok{, }\AttributeTok{y =} \ConstantTok{NULL}\NormalTok{)}

\NormalTok{(v1 }\SpecialCharTok{+}\NormalTok{ v2 }\SpecialCharTok{+}\NormalTok{ v3)}
\end{Highlighting}
\end{Shaded}

\includegraphics{pokemon_files/figure-latex/unnamed-chunk-2-1.pdf}

We also looked at how the Pokemon primary type effects it's happiness.
Looking at the mosaic plot below we can see that there are different
distributions of happiness bases upon primary type. Some observations to
be made is that the most unhappy pokemon types are dragon, dark, and
steel. Also the majority of Pokemon appear to be neutral.

\begin{Shaded}
\begin{Highlighting}[]
\NormalTok{m1 }\OtherTok{\textless{}{-}}\NormalTok{ pokemon\_new }\SpecialCharTok{|}\ErrorTok{\textgreater{}} \FunctionTok{ggplot}\NormalTok{() }\SpecialCharTok{+}
  \FunctionTok{geom\_mosaic}\NormalTok{(}\FunctionTok{aes}\NormalTok{(}\AttributeTok{x =} \FunctionTok{product}\NormalTok{(happiness, primary), }\AttributeTok{fill =}\NormalTok{ happiness)) }\SpecialCharTok{+} 
  \FunctionTok{labs}\NormalTok{(}\AttributeTok{title =} \StringTok{"Pokemon Happiness by Primary Type"}\NormalTok{, }\AttributeTok{x =} \StringTok{"Primary Type"}\NormalTok{, }
       \AttributeTok{y =} \StringTok{"Happiness"}\NormalTok{) }\SpecialCharTok{+}
  \FunctionTok{theme}\NormalTok{(}
    \AttributeTok{panel.grid.major.y =} \FunctionTok{element\_blank}\NormalTok{(),}
    \AttributeTok{axis.text.y =} \FunctionTok{element\_blank}\NormalTok{(),}
    \AttributeTok{axis.ticks.y =} \FunctionTok{element\_blank}\NormalTok{(),}
    \AttributeTok{axis.text.x.bottom =} \FunctionTok{element\_text}\NormalTok{(}\AttributeTok{angle =} \DecValTok{60}\NormalTok{, }\AttributeTok{vjust =} \DecValTok{1}\NormalTok{, }\AttributeTok{hjust =} \DecValTok{1}\NormalTok{)}
\NormalTok{  )}

\NormalTok{m1}
\end{Highlighting}
\end{Shaded}

\includegraphics{pokemon_files/figure-latex/unnamed-chunk-3-1.pdf}

We were also curious if the female rate effected happiness of the
Pokemon. Converting the happiness variable to a likert scale we can
easily see the amount of data points in each bin of female rate in the
following plot. As seen in the plots before there is a smaller amount of
Pokemon with a high female rate. Notice how it appears that depending of
the female rate there are different levels of happiness that are
observed. What stood out to us what that Pokemon with a female rate of
0.25 or 0.875 are only observed with neutral happiness and there are no
very unhappy Pokemon with a rate of 1. There is no apparent correlation
between happiness and female rate.

\begin{Shaded}
\begin{Highlighting}[]
\NormalTok{pokemon\_new }\SpecialCharTok{|}\ErrorTok{\textgreater{}} \FunctionTok{ggplot}\NormalTok{(}\FunctionTok{aes}\NormalTok{(}\FunctionTok{as.factor}\NormalTok{(female\_rate), happiness)) }\SpecialCharTok{+}
  \FunctionTok{geom\_bin2d}\NormalTok{(}\AttributeTok{alpha =}\NormalTok{ .}\DecValTok{9}\NormalTok{) }\SpecialCharTok{+}
  \FunctionTok{scale\_fill\_viridis\_c}\NormalTok{() }\SpecialCharTok{+}
  \FunctionTok{labs}\NormalTok{(}\AttributeTok{title =} \StringTok{"Happiness of Pokemon by Female Rate"}\NormalTok{, }\AttributeTok{x =} \StringTok{"Female Rate"}\NormalTok{, }\AttributeTok{y =} \StringTok{"Happiness"}\NormalTok{) }\SpecialCharTok{+}
  \FunctionTok{theme}\NormalTok{(}\AttributeTok{panel.grid.major.x =} \FunctionTok{element\_blank}\NormalTok{())}
\end{Highlighting}
\end{Shaded}

\includegraphics{pokemon_files/figure-latex/unnamed-chunk-4-1.pdf}

\end{document}
